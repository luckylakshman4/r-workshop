% Options for packages loaded elsewhere
\PassOptionsToPackage{unicode}{hyperref}
\PassOptionsToPackage{hyphens}{url}
%
\documentclass[
]{article}
\usepackage{amsmath,amssymb}
\usepackage{lmodern}
\usepackage{iftex}
\ifPDFTeX
  \usepackage[T1]{fontenc}
  \usepackage[utf8]{inputenc}
  \usepackage{textcomp} % provide euro and other symbols
\else % if luatex or xetex
  \usepackage{unicode-math}
  \defaultfontfeatures{Scale=MatchLowercase}
  \defaultfontfeatures[\rmfamily]{Ligatures=TeX,Scale=1}
\fi
% Use upquote if available, for straight quotes in verbatim environments
\IfFileExists{upquote.sty}{\usepackage{upquote}}{}
\IfFileExists{microtype.sty}{% use microtype if available
  \usepackage[]{microtype}
  \UseMicrotypeSet[protrusion]{basicmath} % disable protrusion for tt fonts
}{}
\makeatletter
\@ifundefined{KOMAClassName}{% if non-KOMA class
  \IfFileExists{parskip.sty}{%
    \usepackage{parskip}
  }{% else
    \setlength{\parindent}{0pt}
    \setlength{\parskip}{6pt plus 2pt minus 1pt}}
}{% if KOMA class
  \KOMAoptions{parskip=half}}
\makeatother
\usepackage{xcolor}
\usepackage[margin=1in]{geometry}
\usepackage{color}
\usepackage{fancyvrb}
\newcommand{\VerbBar}{|}
\newcommand{\VERB}{\Verb[commandchars=\\\{\}]}
\DefineVerbatimEnvironment{Highlighting}{Verbatim}{commandchars=\\\{\}}
% Add ',fontsize=\small' for more characters per line
\usepackage{framed}
\definecolor{shadecolor}{RGB}{248,248,248}
\newenvironment{Shaded}{\begin{snugshade}}{\end{snugshade}}
\newcommand{\AlertTok}[1]{\textcolor[rgb]{0.94,0.16,0.16}{#1}}
\newcommand{\AnnotationTok}[1]{\textcolor[rgb]{0.56,0.35,0.01}{\textbf{\textit{#1}}}}
\newcommand{\AttributeTok}[1]{\textcolor[rgb]{0.77,0.63,0.00}{#1}}
\newcommand{\BaseNTok}[1]{\textcolor[rgb]{0.00,0.00,0.81}{#1}}
\newcommand{\BuiltInTok}[1]{#1}
\newcommand{\CharTok}[1]{\textcolor[rgb]{0.31,0.60,0.02}{#1}}
\newcommand{\CommentTok}[1]{\textcolor[rgb]{0.56,0.35,0.01}{\textit{#1}}}
\newcommand{\CommentVarTok}[1]{\textcolor[rgb]{0.56,0.35,0.01}{\textbf{\textit{#1}}}}
\newcommand{\ConstantTok}[1]{\textcolor[rgb]{0.00,0.00,0.00}{#1}}
\newcommand{\ControlFlowTok}[1]{\textcolor[rgb]{0.13,0.29,0.53}{\textbf{#1}}}
\newcommand{\DataTypeTok}[1]{\textcolor[rgb]{0.13,0.29,0.53}{#1}}
\newcommand{\DecValTok}[1]{\textcolor[rgb]{0.00,0.00,0.81}{#1}}
\newcommand{\DocumentationTok}[1]{\textcolor[rgb]{0.56,0.35,0.01}{\textbf{\textit{#1}}}}
\newcommand{\ErrorTok}[1]{\textcolor[rgb]{0.64,0.00,0.00}{\textbf{#1}}}
\newcommand{\ExtensionTok}[1]{#1}
\newcommand{\FloatTok}[1]{\textcolor[rgb]{0.00,0.00,0.81}{#1}}
\newcommand{\FunctionTok}[1]{\textcolor[rgb]{0.00,0.00,0.00}{#1}}
\newcommand{\ImportTok}[1]{#1}
\newcommand{\InformationTok}[1]{\textcolor[rgb]{0.56,0.35,0.01}{\textbf{\textit{#1}}}}
\newcommand{\KeywordTok}[1]{\textcolor[rgb]{0.13,0.29,0.53}{\textbf{#1}}}
\newcommand{\NormalTok}[1]{#1}
\newcommand{\OperatorTok}[1]{\textcolor[rgb]{0.81,0.36,0.00}{\textbf{#1}}}
\newcommand{\OtherTok}[1]{\textcolor[rgb]{0.56,0.35,0.01}{#1}}
\newcommand{\PreprocessorTok}[1]{\textcolor[rgb]{0.56,0.35,0.01}{\textit{#1}}}
\newcommand{\RegionMarkerTok}[1]{#1}
\newcommand{\SpecialCharTok}[1]{\textcolor[rgb]{0.00,0.00,0.00}{#1}}
\newcommand{\SpecialStringTok}[1]{\textcolor[rgb]{0.31,0.60,0.02}{#1}}
\newcommand{\StringTok}[1]{\textcolor[rgb]{0.31,0.60,0.02}{#1}}
\newcommand{\VariableTok}[1]{\textcolor[rgb]{0.00,0.00,0.00}{#1}}
\newcommand{\VerbatimStringTok}[1]{\textcolor[rgb]{0.31,0.60,0.02}{#1}}
\newcommand{\WarningTok}[1]{\textcolor[rgb]{0.56,0.35,0.01}{\textbf{\textit{#1}}}}
\usepackage{graphicx}
\makeatletter
\def\maxwidth{\ifdim\Gin@nat@width>\linewidth\linewidth\else\Gin@nat@width\fi}
\def\maxheight{\ifdim\Gin@nat@height>\textheight\textheight\else\Gin@nat@height\fi}
\makeatother
% Scale images if necessary, so that they will not overflow the page
% margins by default, and it is still possible to overwrite the defaults
% using explicit options in \includegraphics[width, height, ...]{}
\setkeys{Gin}{width=\maxwidth,height=\maxheight,keepaspectratio}
% Set default figure placement to htbp
\makeatletter
\def\fps@figure{htbp}
\makeatother
\setlength{\emergencystretch}{3em} % prevent overfull lines
\providecommand{\tightlist}{%
  \setlength{\itemsep}{0pt}\setlength{\parskip}{0pt}}
\setcounter{secnumdepth}{-\maxdimen} % remove section numbering
\usepackage{booktabs}
\usepackage{longtable}
\usepackage{array}
\usepackage{multirow}
\usepackage{wrapfig}
\usepackage{float}
\usepackage{colortbl}
\usepackage{pdflscape}
\usepackage{tabu}
\usepackage{threeparttable}
\usepackage{threeparttablex}
\usepackage[normalem]{ulem}
\usepackage{makecell}
\usepackage{xcolor}
\ifLuaTeX
  \usepackage{selnolig}  % disable illegal ligatures
\fi
\IfFileExists{bookmark.sty}{\usepackage{bookmark}}{\usepackage{hyperref}}
\IfFileExists{xurl.sty}{\usepackage{xurl}}{} % add URL line breaks if available
\urlstyle{same} % disable monospaced font for URLs
\hypersetup{
  pdftitle={pivot\_lobsters},
  pdfauthor={Lakshman},
  hidelinks,
  pdfcreator={LaTeX via pandoc}}

\title{pivot\_lobsters}
\author{Lakshman}
\date{2022-12-02}

\begin{document}
\maketitle

\hypertarget{pivot-tables-with-dplyr}{%
\section{Pivot Tables with dplyr}\label{pivot-tables-with-dplyr}}

create these tables using the group\_by and summarize functions from the
dplyr package (part of the Tidyverse)

\begin{Shaded}
\begin{Highlighting}[]
\DocumentationTok{\#\# attach libraries}
\FunctionTok{library}\NormalTok{(tidyverse)}
\end{Highlighting}
\end{Shaded}

\begin{verbatim}
## -- Attaching packages --------------------------------------- tidyverse 1.3.2 --
## v ggplot2 3.4.0      v purrr   0.3.5 
## v tibble  3.1.8      v dplyr   1.0.10
## v tidyr   1.2.1      v stringr 1.4.1 
## v readr   2.1.3      v forcats 0.5.2 
## -- Conflicts ------------------------------------------ tidyverse_conflicts() --
## x dplyr::filter() masks stats::filter()
## x dplyr::lag()    masks stats::lag()
\end{verbatim}

\begin{Shaded}
\begin{Highlighting}[]
\FunctionTok{library}\NormalTok{(readxl)}
\FunctionTok{library}\NormalTok{(here)}
\end{Highlighting}
\end{Shaded}

\begin{verbatim}
## here() starts at D:/New folder/R/WD/r-for-excel-data/r-workshop/r-workshop
\end{verbatim}

\begin{Shaded}
\begin{Highlighting}[]
\FunctionTok{library}\NormalTok{(skimr) }\CommentTok{\# install.packages(\textquotesingle{}skimr\textquotesingle{})}
\FunctionTok{library}\NormalTok{(kableExtra) }\CommentTok{\# install.packages(\textquotesingle{}kableExtra\textquotesingle{})}
\end{Highlighting}
\end{Shaded}

\begin{verbatim}
## Warning in !is.null(rmarkdown::metadata$output) && rmarkdown::metadata$output
## %in% : 'length(x) = 2 > 1' in coercion to 'logical(1)'
\end{verbatim}

\begin{verbatim}
## 
## Attaching package: 'kableExtra'
## 
## The following object is masked from 'package:dplyr':
## 
##     group_rows
\end{verbatim}

\hypertarget{read-in-data}{%
\section{Read in data}\label{read-in-data}}

\begin{Shaded}
\begin{Highlighting}[]
\NormalTok{lobsters }\OtherTok{\textless{}{-}} \FunctionTok{read\_xlsx}\NormalTok{(}\FunctionTok{here}\NormalTok{(}\StringTok{"data/lobsters.xlsx"}\NormalTok{), }\AttributeTok{skip=}\DecValTok{4}\NormalTok{)}
\end{Highlighting}
\end{Shaded}

use head(lobsters) \# for top 6 rows

\hypertarget{explore-data}{%
\subsection{explore data}\label{explore-data}}

skimr::skim(lobsters) \# skim lets us look more at each variable

\begin{Shaded}
\begin{Highlighting}[]
\CommentTok{\# skimr::skim(lobsters) \# skim lets us look more at each variable}
\end{Highlighting}
\end{Shaded}

group\_by() \%\textgreater\% summarize() In R, we can create the
functionality of pivot tables with the same logic: we will tell R to
group by something and then summarize by something

Take the data and then group by something and then summarize by
something Syntax data \%\textgreater\% group\_by() \%\textgreater\%
summarize()

The pipe operator \%\textgreater\% is a really critical feature of the
dplyr package, originally created for the magrittr package. It lets us
chain together steps of our data wrangling, enabling us to tell a clear
story about our entire data analysis.

View(lobsters) shows up in your Console. View() (capital V) is the R
function to view any variable in the viewer.

\hypertarget{group_by-one-variable}{%
\section{group\_by one variable}\label{group_by-one-variable}}

group\_by() \%\textgreater\% summarize() with our lobsters data, just
like we did in Excel. We will first group\_by year and then summarize by
count, using the function n() (in the dplyr package). n() counts the
number of times an observation shows up, and since this is uncounted
data, this will count each row.

\begin{Shaded}
\begin{Highlighting}[]
\NormalTok{lobsters }\SpecialCharTok{\%\textgreater{}\%}
\FunctionTok{group\_by}\NormalTok{(year) }\SpecialCharTok{\%\textgreater{}\%}
  \FunctionTok{summarise}\NormalTok{(}\AttributeTok{count\_by\_year=}\FunctionTok{n}\NormalTok{())}
\end{Highlighting}
\end{Shaded}

\begin{verbatim}
## # A tibble: 5 x 2
##    year count_by_year
##   <dbl>         <int>
## 1  2012           231
## 2  2013           243
## 3  2014           510
## 4  2015          1100
## 5  2016           809
\end{verbatim}

\hypertarget{group_by-multiple-variables}{%
\section{group\_by multiple
variables}\label{group_by-multiple-variables}}

\begin{Shaded}
\begin{Highlighting}[]
\NormalTok{lobsters }\SpecialCharTok{\%\textgreater{}\%}
  \FunctionTok{group\_by}\NormalTok{(site, year) }\SpecialCharTok{\%\textgreater{}\%}
  \FunctionTok{summarize}\NormalTok{(}\AttributeTok{count\_by\_siteyear =}  \FunctionTok{n}\NormalTok{())}
\end{Highlighting}
\end{Shaded}

\begin{verbatim}
## `summarise()` has grouped output by 'site'. You can override using the
## `.groups` argument.
\end{verbatim}

\begin{verbatim}
## # A tibble: 25 x 3
## # Groups:   site [5]
##    site   year count_by_siteyear
##    <chr> <dbl>             <int>
##  1 aque   2012                38
##  2 aque   2013                32
##  3 aque   2014               100
##  4 aque   2015                83
##  5 aque   2016                48
##  6 carp   2012                78
##  7 carp   2013                93
##  8 carp   2014                79
##  9 carp   2015                90
## 10 carp   2016               231
## # ... with 15 more rows
\end{verbatim}

\hypertarget{summarize-multiple-variables}{%
\section{summarize multiple
variables}\label{summarize-multiple-variables}}

Let's also calculate the mean and standard deviation. First let's use
the mean() function to calculate the mean. We do this within the same
summarize() function

\begin{Shaded}
\begin{Highlighting}[]
\NormalTok{lobsters }\SpecialCharTok{\%\textgreater{}\%}
  \FunctionTok{group\_by}\NormalTok{(site, year) }\SpecialCharTok{\%\textgreater{}\%}
  \FunctionTok{summarize}\NormalTok{(}\AttributeTok{count\_by\_siteyear =}  \FunctionTok{n}\NormalTok{(),}
            \AttributeTok{mean\_size\_mm =} \FunctionTok{mean}\NormalTok{(size\_mm)) }\CommentTok{\# by count and mean}
\end{Highlighting}
\end{Shaded}

\begin{verbatim}
## `summarise()` has grouped output by 'site'. You can override using the
## `.groups` argument.
\end{verbatim}

\begin{verbatim}
## # A tibble: 25 x 4
## # Groups:   site [5]
##    site   year count_by_siteyear mean_size_mm
##    <chr> <dbl>             <int>        <dbl>
##  1 aque   2012                38         71  
##  2 aque   2013                32         72.1
##  3 aque   2014               100         76.9
##  4 aque   2015                83         68.5
##  5 aque   2016                48         68.7
##  6 carp   2012                78         74.4
##  7 carp   2013                93         76.6
##  8 carp   2014                79         NA  
##  9 carp   2015                90         70.7
## 10 carp   2016               231         68.9
## # ... with 15 more rows
\end{verbatim}

NA because one or more values in that year are NA. pass an argument
na.rm=TRUE that says to remove NAs first before calculating the average.
Then Calculate the standard deviation with the sd() function

\begin{Shaded}
\begin{Highlighting}[]
\NormalTok{lobsters }\SpecialCharTok{\%\textgreater{}\%}
  \FunctionTok{group\_by}\NormalTok{(site, year) }\SpecialCharTok{\%\textgreater{}\%}
  \FunctionTok{summarize}\NormalTok{(}\AttributeTok{count\_by\_siteyear =}  \FunctionTok{n}\NormalTok{(), }
            \AttributeTok{mean\_size\_mm =} \FunctionTok{mean}\NormalTok{(size\_mm, }\AttributeTok{na.rm=}\ConstantTok{TRUE}\NormalTok{), }\CommentTok{\#na.rm to remove NA}
            \AttributeTok{sd\_size\_mm =} \FunctionTok{sd}\NormalTok{(size\_mm, }\AttributeTok{na.rm=}\ConstantTok{TRUE}\NormalTok{)) }\CommentTok{\#summarise by count, mean, sd}
\end{Highlighting}
\end{Shaded}

\begin{verbatim}
## `summarise()` has grouped output by 'site'. You can override using the
## `.groups` argument.
\end{verbatim}

\begin{verbatim}
## # A tibble: 25 x 5
## # Groups:   site [5]
##    site   year count_by_siteyear mean_size_mm sd_size_mm
##    <chr> <dbl>             <int>        <dbl>      <dbl>
##  1 aque   2012                38         71        10.2 
##  2 aque   2013                32         72.1      12.3 
##  3 aque   2014               100         76.9       9.32
##  4 aque   2015                83         68.5      12.6 
##  5 aque   2016                48         68.7      12.5 
##  6 carp   2012                78         74.4      14.6 
##  7 carp   2013                93         76.6       8.71
##  8 carp   2014                79         79.1       8.57
##  9 carp   2015                90         70.7      14.6 
## 10 carp   2016               231         68.9      12.5 
## # ... with 15 more rows
\end{verbatim}

Now we are at the point where we actually want to save this summary
information as a variable so we can use it in further analyses and
formatting.

So let's add a variable assignment to that first line:

\begin{Shaded}
\begin{Highlighting}[]
\NormalTok{siteyear\_summary }\OtherTok{\textless{}{-}}\NormalTok{ lobsters }\SpecialCharTok{\%\textgreater{}\%}
  \FunctionTok{group\_by}\NormalTok{(site, year) }\SpecialCharTok{\%\textgreater{}\%}
  \FunctionTok{summarize}\NormalTok{(}\AttributeTok{count\_by\_siteyear =}  \FunctionTok{n}\NormalTok{(), }
            \AttributeTok{mean\_size\_mm =} \FunctionTok{mean}\NormalTok{(size\_mm, }\AttributeTok{na.rm=}\ConstantTok{TRUE}\NormalTok{), }\CommentTok{\#na.rm to remove NA}
            \AttributeTok{sd\_size\_mm =} \FunctionTok{sd}\NormalTok{(size\_mm, }\AttributeTok{na.rm=}\ConstantTok{TRUE}\NormalTok{)) }\CommentTok{\#summarise by count, mean, sd}
\end{Highlighting}
\end{Shaded}

\begin{verbatim}
## `summarise()` has grouped output by 'site'. You can override using the
## `.groups` argument.
\end{verbatim}

\begin{Shaded}
\begin{Highlighting}[]
\NormalTok{siteyear\_summary }\CommentTok{\#inspect our new variable}
\end{Highlighting}
\end{Shaded}

\begin{verbatim}
## # A tibble: 25 x 5
## # Groups:   site [5]
##    site   year count_by_siteyear mean_size_mm sd_size_mm
##    <chr> <dbl>             <int>        <dbl>      <dbl>
##  1 aque   2012                38         71        10.2 
##  2 aque   2013                32         72.1      12.3 
##  3 aque   2014               100         76.9       9.32
##  4 aque   2015                83         68.5      12.6 
##  5 aque   2016                48         68.7      12.5 
##  6 carp   2012                78         74.4      14.6 
##  7 carp   2013                93         76.6       8.71
##  8 carp   2014                79         79.1       8.57
##  9 carp   2015                90         70.7      14.6 
## 10 carp   2016               231         68.9      12.5 
## # ... with 15 more rows
\end{verbatim}

\hypertarget{table-formatting-with-kable}{%
\section{Table formatting with
kable()}\label{table-formatting-with-kable}}

\begin{Shaded}
\begin{Highlighting}[]
\DocumentationTok{\#\# make a table with our new variable}
\NormalTok{siteyear\_summary }\SpecialCharTok{\%\textgreater{}\%}
  \FunctionTok{kable}\NormalTok{()}
\end{Highlighting}
\end{Shaded}

\begin{tabular}{l|r|r|r|r}
\hline
site & year & count\_by\_siteyear & mean\_size\_mm & sd\_size\_mm\\
\hline
aque & 2012 & 38 & 71.00000 & 10.150223\\
\hline
aque & 2013 & 32 & 72.12500 & 12.262584\\
\hline
aque & 2014 & 100 & 76.92000 & 9.321074\\
\hline
aque & 2015 & 83 & 68.45783 & 12.555536\\
\hline
aque & 2016 & 48 & 68.68750 & 12.510687\\
\hline
carp & 2012 & 78 & 74.35897 & 14.616282\\
\hline
carp & 2013 & 93 & 76.56989 & 8.709562\\
\hline
carp & 2014 & 79 & 79.08974 & 8.569329\\
\hline
carp & 2015 & 90 & 70.65556 & 14.646517\\
\hline
carp & 2016 & 231 & 68.90476 & 12.470122\\
\hline
ivee & 2012 & 26 & 66.07692 & 12.092719\\
\hline
ivee & 2013 & 40 & 73.77500 & 7.640941\\
\hline
ivee & 2014 & 132 & 76.02273 & 17.860984\\
\hline
ivee & 2015 & 361 & 69.80332 & 17.470534\\
\hline
ivee & 2016 & 193 & 71.61658 & 13.450454\\
\hline
mohk & 2012 & 83 & 77.25301 & 10.587433\\
\hline
mohk & 2013 & 15 & 71.86667 & 10.190098\\
\hline
mohk & 2014 & 36 & 75.75000 & 10.038142\\
\hline
mohk & 2015 & 296 & 59.19932 & 16.770357\\
\hline
mohk & 2016 & 210 & 63.01286 & 11.875763\\
\hline
napl & 2012 & 6 & 73.00000 & 11.747340\\
\hline
napl & 2013 & 63 & 75.31746 & 12.989854\\
\hline
napl & 2014 & 163 & 79.51572 & 9.556531\\
\hline
napl & 2015 & 270 & 78.24074 & 12.438899\\
\hline
napl & 2016 & 127 & 74.39370 & 10.732060\\
\hline
\end{tabular}

include median

\begin{Shaded}
\begin{Highlighting}[]
\NormalTok{siteyear\_summary }\OtherTok{\textless{}{-}}\NormalTok{ lobsters }\SpecialCharTok{\%\textgreater{}\%}
  \FunctionTok{group\_by}\NormalTok{(site, year) }\SpecialCharTok{\%\textgreater{}\%}
  \FunctionTok{summarize}\NormalTok{(}\AttributeTok{count\_by\_siteyear =}  \FunctionTok{n}\NormalTok{(), }
            \AttributeTok{mean\_size\_mm =} \FunctionTok{mean}\NormalTok{(size\_mm, }\AttributeTok{na.rm =} \ConstantTok{TRUE}\NormalTok{), }
            \AttributeTok{sd\_size\_mm =} \FunctionTok{sd}\NormalTok{(size\_mm, }\AttributeTok{na.rm =} \ConstantTok{TRUE}\NormalTok{), }
            \AttributeTok{median\_size\_mm =} \FunctionTok{median}\NormalTok{(size\_mm, }\AttributeTok{na.rm =} \ConstantTok{TRUE}\NormalTok{))}
\end{Highlighting}
\end{Shaded}

\begin{verbatim}
## `summarise()` has grouped output by 'site'. You can override using the
## `.groups` argument.
\end{verbatim}

\begin{Shaded}
\begin{Highlighting}[]
\NormalTok{siteyear\_summary}
\end{Highlighting}
\end{Shaded}

\begin{verbatim}
## # A tibble: 25 x 6
## # Groups:   site [5]
##    site   year count_by_siteyear mean_size_mm sd_size_mm median_size_mm
##    <chr> <dbl>             <int>        <dbl>      <dbl>          <dbl>
##  1 aque   2012                38         71        10.2            70  
##  2 aque   2013                32         72.1      12.3            75  
##  3 aque   2014               100         76.9       9.32           75.5
##  4 aque   2015                83         68.5      12.6            70  
##  5 aque   2016                48         68.7      12.5            71  
##  6 carp   2012                78         74.4      14.6            74.5
##  7 carp   2013                93         76.6       8.71           76  
##  8 carp   2014                79         79.1       8.57           79  
##  9 carp   2015                90         70.7      14.6            70  
## 10 carp   2016               231         68.9      12.5            70  
## # ... with 15 more rows
\end{verbatim}

ggplot function

\begin{Shaded}
\begin{Highlighting}[]
\DocumentationTok{\#\# a ggplot option:}
\FunctionTok{ggplot}\NormalTok{(}\AttributeTok{data =}\NormalTok{ siteyear\_summary, }\FunctionTok{aes}\NormalTok{(}\AttributeTok{x =}\NormalTok{ year, }\AttributeTok{y =}\NormalTok{ median\_size\_mm, }\AttributeTok{color =}\NormalTok{ site)) }\SpecialCharTok{+}
  \FunctionTok{geom\_line}\NormalTok{() }\SpecialCharTok{+}
  \FunctionTok{theme\_minimal}\NormalTok{()}
\end{Highlighting}
\end{Shaded}

\includegraphics{pivot_lobsters_files/figure-latex/unnamed-chunk-11-1.pdf}

\begin{Shaded}
\begin{Highlighting}[]
\FunctionTok{ggsave}\NormalTok{(}\FunctionTok{here}\NormalTok{(}\StringTok{"figures"}\NormalTok{, }\StringTok{"lobsters{-}line.png"}\NormalTok{)) }\CommentTok{\# save image}
\end{Highlighting}
\end{Shaded}

\begin{verbatim}
## Saving 6.5 x 4.5 in image
\end{verbatim}

\begin{Shaded}
\begin{Highlighting}[]
\DocumentationTok{\#\# another option:}
\FunctionTok{ggplot}\NormalTok{(siteyear\_summary, }\FunctionTok{aes}\NormalTok{(}\AttributeTok{x =}\NormalTok{ year, }\AttributeTok{y =}\NormalTok{ median\_size\_mm, }\AttributeTok{fill =}\NormalTok{ site, }\AttributeTok{color =}\NormalTok{site)) }\SpecialCharTok{+}
  \FunctionTok{geom\_col}\NormalTok{() }\SpecialCharTok{+}
  \FunctionTok{facet\_wrap}\NormalTok{(}\SpecialCharTok{\textasciitilde{}}\NormalTok{site)}
\end{Highlighting}
\end{Shaded}

\includegraphics{pivot_lobsters_files/figure-latex/unnamed-chunk-12-1.pdf}

\begin{Shaded}
\begin{Highlighting}[]
\FunctionTok{ggsave}\NormalTok{(}\FunctionTok{here}\NormalTok{(}\StringTok{"figures"}\NormalTok{, }\StringTok{"lobsters{-}col.png"}\NormalTok{))}
\end{Highlighting}
\end{Shaded}

\begin{verbatim}
## Saving 6.5 x 4.5 in image
\end{verbatim}

\#dplyr::count()

Now that we've spent time with group\_by \%\textgreater\% summarize,
there is a shortcut if you only want to summarize by count. This is with
a function called count(), and it will group\_by your selected variable,
count, and then also ungroup.

\begin{Shaded}
\begin{Highlighting}[]
\NormalTok{lobsters }\SpecialCharTok{\%\textgreater{}\%}
  \FunctionTok{count}\NormalTok{(site, year)}
\end{Highlighting}
\end{Shaded}

\begin{verbatim}
## # A tibble: 25 x 3
##    site   year     n
##    <chr> <dbl> <int>
##  1 aque   2012    38
##  2 aque   2013    32
##  3 aque   2014   100
##  4 aque   2015    83
##  5 aque   2016    48
##  6 carp   2012    78
##  7 carp   2013    93
##  8 carp   2014    79
##  9 carp   2015    90
## 10 carp   2016   231
## # ... with 15 more rows
\end{verbatim}

\begin{Shaded}
\begin{Highlighting}[]
\DocumentationTok{\#\# This is the same as:}
\NormalTok{lobsters }\SpecialCharTok{\%\textgreater{}\%}
  \FunctionTok{group\_by}\NormalTok{(site, year) }\SpecialCharTok{\%\textgreater{}\%} 
  \FunctionTok{summarize}\NormalTok{(}\AttributeTok{n =} \FunctionTok{n}\NormalTok{()) }\SpecialCharTok{\%\textgreater{}\%}
  \FunctionTok{ungroup}\NormalTok{()}
\end{Highlighting}
\end{Shaded}

\begin{verbatim}
## `summarise()` has grouped output by 'site'. You can override using the
## `.groups` argument.
\end{verbatim}

\begin{verbatim}
## # A tibble: 25 x 3
##    site   year     n
##    <chr> <dbl> <int>
##  1 aque   2012    38
##  2 aque   2013    32
##  3 aque   2014   100
##  4 aque   2015    83
##  5 aque   2016    48
##  6 carp   2012    78
##  7 carp   2013    93
##  8 carp   2014    79
##  9 carp   2015    90
## 10 carp   2016   231
## # ... with 15 more rows
\end{verbatim}

\hypertarget{make-new-variable-with-mutate}{%
\section{Make new variable with
mutate()}\label{make-new-variable-with-mutate}}

The sizes are in millimeters but let's say it was important for them to
be in meters. We can add a column with this calculation

\begin{Shaded}
\begin{Highlighting}[]
\NormalTok{lobsters }\SpecialCharTok{\%\textgreater{}\%}
  \FunctionTok{mutate}\NormalTok{(}\AttributeTok{size\_m =}\NormalTok{ size\_mm }\SpecialCharTok{/} \DecValTok{1000}\NormalTok{)}
\end{Highlighting}
\end{Shaded}

\begin{verbatim}
## # A tibble: 2,893 x 8
##     year month date    site  transect replicate size_mm size_m
##    <dbl> <dbl> <chr>   <chr>    <dbl> <chr>       <dbl>  <dbl>
##  1  2012     8 8/20/12 ivee         3 A              70  0.07 
##  2  2012     8 8/20/12 ivee         3 B              60  0.06 
##  3  2012     8 8/20/12 ivee         3 B              65  0.065
##  4  2012     8 8/20/12 ivee         3 B              70  0.07 
##  5  2012     8 8/20/12 ivee         3 B              85  0.085
##  6  2012     8 8/20/12 ivee         3 C              60  0.06 
##  7  2012     8 8/20/12 ivee         3 C              65  0.065
##  8  2012     8 8/20/12 ivee         3 C              67  0.067
##  9  2012     8 8/20/12 ivee         3 D              70  0.07 
## 10  2012     8 8/20/12 ivee         4 B              85  0.085
## # ... with 2,883 more rows
\end{verbatim}

If we want to add a column that has the same value repeated, we can pass
it just one value, either a number or a character string (in quotes).
And let's save this as a variable called lobsters\_detailed

\begin{Shaded}
\begin{Highlighting}[]
\NormalTok{lobsters\_detailed }\OtherTok{\textless{}{-}}\NormalTok{ lobsters }\SpecialCharTok{\%\textgreater{}\%}
  \FunctionTok{mutate}\NormalTok{(}\AttributeTok{size\_m =}\NormalTok{ size\_mm }\SpecialCharTok{/} \DecValTok{1000}\NormalTok{, }
         \AttributeTok{millenia =} \DecValTok{2000}\NormalTok{,}
         \AttributeTok{observer =} \StringTok{"Allison Horst"}\NormalTok{)}
\end{Highlighting}
\end{Shaded}

\hypertarget{select}{%
\section{select()}\label{select}}

To choose, retain, and move your data by columns To present this data
finally with only columns for date, site, and size in meters

\begin{Shaded}
\begin{Highlighting}[]
\NormalTok{lobsters\_detailed }\SpecialCharTok{\%\textgreater{}\%}
  \FunctionTok{select}\NormalTok{(date, site, size\_m)}
\end{Highlighting}
\end{Shaded}

\begin{verbatim}
## # A tibble: 2,893 x 3
##    date    site  size_m
##    <chr>   <chr>  <dbl>
##  1 8/20/12 ivee   0.07 
##  2 8/20/12 ivee   0.06 
##  3 8/20/12 ivee   0.065
##  4 8/20/12 ivee   0.07 
##  5 8/20/12 ivee   0.085
##  6 8/20/12 ivee   0.06 
##  7 8/20/12 ivee   0.065
##  8 8/20/12 ivee   0.067
##  9 8/20/12 ivee   0.07 
## 10 8/20/12 ivee   0.085
## # ... with 2,883 more rows
\end{verbatim}

\end{document}
